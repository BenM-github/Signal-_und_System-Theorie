\documentclass{article}
\usepackage[german]{babel}
\usepackage[letterpaper,top=2cm,bottom=2cm,left=3cm,right=3cm,marginparwidth=1.75cm]{geometry}

% Useful packages
\usepackage{amsmath}
\usepackage{graphicx}
\usepackage[colorlinks=true, allcolors=blue]{hyperref}
\usepackage{multicol}

\title{Signal und System Theorie}
\author{Prof. Dr.-Ing. Frank Giesecke}
\date{}

\begin{document}
\maketitle

\section*{28.10.2024}

\subsection*{Wiederholung}

\begin{enumerate}
\item Leistungssignale \\
Gleichanteil/ linearer Mittelwert/ Moment 1. Ordnung \\
$\Bar{x} = (coming_soon)$
\[
    f(x) = \lim_{T\to\infty} \int_{-T}^{T} x^2 \,dt 
\]

\item Signalgleichleistung \\
(Leistung, die den Gleichanteil verursacht) \\
$P_{x=} = \Bar{x}^2 = m_1^2$

\item mittlere Signalleistung/ quadratische Mittelwert/ Gesamt-Signalleistung / Moment 2. Ordnung \\
$P = \Bar{x^2} = m_2 = (coming_soon)$

\item Effektivwert (Wurzel aus der mittleren Signalleistung) \\
$x_{eff}=\sqrt{P}=\sqrt{x^2}$

\item Signalwechselleistung/ Varianz/ Zentral-Moment 2. Ordnung \\
$P_{x~}=\sigma^2 = \mu_2 = P_x - P_{x=} = \Bar{x^2} - \Bar{x}^2$

\item Standartabweichung \\
$\sigma = \sqrt{P_{x~}} = \sqrt{\mu_2}$
\end{enumerate}

\subsection*{Alternative Berechnung der Signalwechselleistung/ Varianz/ Zentral-Moment 2. Ordnung}
$P_{x~} = (coming_soon)$ \\
$P_{x~} = P_x - P_x$

\subsection*{Direkte Berechnung der Varianz/ Signalwechselspannung aus einem Datensatz (digital)}

\begin{center}
    Fallunterscheidung
\end{center}
\begin{multicols}{2}
    linerer Mittelwert/ Gleichanteil ist bekannt oder kann exakt bestimmt werden. \\
    $\Bar{x} ist bekannt$ \\
    $\sigma^2 = P_{x~} = \mu_2 = (coming_soon)$
    
    \columnbreak
    linearer Mittelwert wird aus den N Werten bestimmt. \\
    $\Bar{x} ist un-bekannt$ \\
    $\Bar{x_N} = (coming_soon)$ \\
    $\sigma^2 = P_x = (coming_soon)$
    
\end{multicols}
\begin{center}
    N = Anzahl der Werte
\end{center}

\subsection*{Es folgt:}
Angenäherter Einheitssprung, Einheitssprung-Funktion, Einheitsimpuls-Funktion, Deltaimpuls/ Dirac-Impuls, Einheitsanstiegs-Funktion

\subsection*{Angenäherter Einheitssprung ($\delta$ delta)}
\begin{multicols}{3}
    (grafic is coming soon)

    \columnbreak

    \begin{center}
    $\overrightarrow{Differentation}$ \\
    $(\frac{d}{dt})$ \\
    \vspace{1em}
    $\overleftarrow{Integration}$ \\
    $\hookrightarrow$ Intigrationsgrenzen \\
    $-\infty bis aktueller Zeitpunkt (t)$ \\
    $\sigma = (coming_soon)$
    \end{center}

    \columnbreak

    (grafic is coming soon)
    
    
\end{multicols}

\section*{\cen  tering **.**.2024 Vorlesung noch nicht nachgetragen}

\newpage
\section*{\centering 11.11.2024}
\subsection*{\centering Ergänzung: Kreuzrelation}

\[
    E_{x1x2}(\tau)=\int_{-\infty}^{\infty}x(t)*x_2(t+\tau)*dt
\]
mindestens eines der Verläufe muss ein Energiesignal sein. \\
Wenn beide Verläufe $x_1$ und $x_2$ Leistungssignale sind, dann:
\[
    Allg. Variante: P_{x1x2}(\tau)=\lim_{T\to\infty}x_1(t)x_2*x_2(t+\tau)*dt
\]

    bei Vorliegen einer Periodizität von $x_1$ und $x_2$
\[
    P_{x1x2}(\tau)=\frac{1}{T}\int_{0}^{T}x(t)*x_2(t+\tau)*dt
\]
entweder T als gleiche Periode bei den Verläufen oder T als gemeinsames Vielfaches der beiden Periodenduaer von $x_1$ und $x_2$

\subsection*{\centering Das System}
(system bild fehlt noch wurde aber schon erstellt)

\subsubsection*{\centering Im Zeitbereich:}
z.B. Spannung $u_e(t) \to u_a(t)$ \\
oder digital/zeitdiskret

\subsubsection*{\centering System im Laplace Bereich}
$s=\sigma+j\omega$ \\
(system im laplace-bereich bild fehlt wurde aber schon erstellt)

\subsubsection*{\centering System in Frequenzbereich}

\subsubsection*{\centering System-Eigenschaften}
4 Grundeigenschaften:
\begin{itemize}
    \item Linearität
    \begin{itemize}
        \item linear
        \item licht linear
    \end{itemize}
    
    \item Zeitinvarianz
    \begin{itemize}
        \item Zeit-konstant
        \item Zeit-veränderlich
    \end{itemize}
    
    \item Kausalität
    \begin{itemize}
        \item kausal
        \begin{itemize}
            \item statisch (speicherlos)
            \item dynamisch (mit Speicherelementen)
        \end{itemize}
        \item a-kausal (nicht kausal)
    \end{itemize}
    
    \item Stabilität
    \begin{itemize}
        \item Stabil
        \item Grenz-Stabil
        \item In-Stabil
    \end{itemize}
\end{itemize}


\end{document}